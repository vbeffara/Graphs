\chapter{Graph Minors}

\section{Well-quasi-ordering}

\begin{proposition}
  \label{WQO_iff} \lean{WQO_iff} \leanok \mathlibok
  A quasi-ordering $\leq$ on $X$ is a well-quasi-ordering if and only if $X$
  contains neither an infinite antichain nor an infinite strictly decreasing
  sequence $x_0 > x_1 > \cdots$.
\end{proposition}

\begin{proof}
  \leanok
  Diestel's book uses Ramsey theory to prove this but in fact it is essentially
  in mathlib already.
\end{proof}

\begin{corollary}
  \label{WQO_incsubseq} \uses{WQO_iff}
  If $X$ is well-quasi-ordered, then every infinite sequence in $X$ has an
  infinite increasing subsequence.
\end{corollary}

\begin{proof}
  Let \(\leq\) be a quasi-order on \(X\). 
  
  Suppose that \(X\) is well-quasi-ordered.

  Let \((x_i)_{i\in\mathbb{N}}\) an infinite sequence in \(X\).

  There exists an index \(i\) such that \(\{j>i/x_i\leq x_j\}\) is infinite.

  Indeed, if we suppose not, for all index \(i\), there exists \(j>i\) such that \(x_i\not\leq x_j\)

  Therefore, as \((x_i)_{i\in\mathbb{N}}\) is infinite, we can construct an infinite subsequence \((x_{i_k})_{k\in\mathbb{N}}\) such that \(\forall k\in\mathbb{N}, x_{i_k}>\x_{i_{k+1}}\) or \(x_{i_k}\) and \(\x_{i_{k+1}}\) are incomparable
  
  Suppose there is an infinite subsequence of incomparable elements, this is an antichain which contradicts the well-quasi-ordered property.

  If not, there exists an index \(k_*\) such that \(\forall k>k_*, x_{i_k}>\x_{i_{k+1}}\), this is an infinite strictly decreasing sequence which also contradicts the well-quasi-ordered property.

  Let \(i_0\) be an index such that \(\{j>i_0/x_{i_0}\leq x_j\}\) is infinite.

  We proceed by induction. 
  
  Let \(n\in\mathbb{N}\), we suppose constructed \(x_{i_0},...,x_{i_n}\) such that \(i_0<...<i_n\), \(x_{i_0}\leq ...\leq x_{i_n}\) and \(A_n:=\{j>i_n/x_{i_n}\leq x_j\}\) is infinite.
  
  \((x_i)_{i\in A_n}\) is an infinite sequence in \(X\), therefore, according to the previous result, there exists an index \(i_{n+1}\in A_n\) such that \(A_{n+1}=\{j>i_{n+1}/x_{i_{n+1}}\leq x_j\}\) is infinite.
  
  As \(i_{n+1}\in A_n\), it follows that \(i_n<i_{n+1}\) and \(x_{i_n}\leq x_{i_{n+1}}\)
  
  We have obtained \(x_{i_0},...,x_{i_{n+1}}\) such that \(i_0<...<i_{n+1}\), \(x_{i_0}\leq ...\leq x_{i_{n+1}}\) and \(A_{n+1}=\{j>i_n/x_{i_{n+1}}\leq x_j\}\) is infinite, which concludes the induction
  
  We have constructed an infinite increasing subsequence \((x_{i_k})_{k\in\mathbb{N}}\) of \((x_i)_{i\in\mathbb{N}}\)
\end{proof}

\begin{theorem}[Higman's Lemma, Diestel 12.1.3]
  \label{theo:Higman} \lean{Higman} \leanok
  If $X$ is well-quasi-ordered by , then so is $[X]^{<ω}$.
\end{theorem}

\section{The graph minor theorem for trees}

\setcounter{section}{6}
\section{The graph minor theorem}

\begin{theorem}
  \label{theo:GraphMinorTheorem} \lean{GraphMinorTheorem} \uses{def:minor} \leanok
  The finite graphs are well-quasi-ordered by the minor relation $\preceq$.
\end{theorem}