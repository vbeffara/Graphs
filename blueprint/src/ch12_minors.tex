\chapter{Graph Minors}

\section{Well-quasi-ordering}

\begin{proposition}
  \label{WQO_iff} \lean{WQO_iff} \leanok \mathlibok
  A quasi-ordering $\leq$ on $X$ is a well-quasi-ordering if and only if $X$
  contains neither an infinite antichain nor an infinite strictly decreasing
  sequence $x_0 > x_1 > \cdots$.
\end{proposition}

\begin{proof}
  \leanok
  Diestel's book uses Ramsey theory to prove this but in fact it is essentially
  in mathlib already.
\end{proof}

\begin{corollary}
  \label{WQO_incsubseq} \uses{WQO_iff}
  If $X$ is well-quasi-ordered, then every infinite sequence in $X$ has an
  infinite increasing subsequence.
\end{corollary}

\begin{lemma}
  \label{WQO_finset} \lean{WQO_Finset} \leanok
  If $X$ is well-quasi-ordered by , then so is $[X]^{<ω}$.
\end{lemma}

\section{The graph minor theorem for trees}

\setcounter{section}{6}
\section{The graph minor theorem}

\begin{theorem}
  \label{theo:GraphMinorTheorem} \lean{GraphMinorTheorem} \uses{def:minor} \leanok
  The finite graphs are well-quasi-ordered by the minor relation $\preceq$.
\end{theorem}