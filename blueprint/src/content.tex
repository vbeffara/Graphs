% In this file you should put the actual content of the blueprint.
% It will be used both by the web and the print version.
% It should *not* include the \begin{document}
%
% If you want to split the blueprint content into several files then
% the current file can be a simple sequence of \input. Otherwise It
% can start with a \section or \chapter for instance.

The numbering follows Diestel, \emph{Graph Theory}, 5th edition. In particular we will mostly use the same notation.
\begin{description}
  \item[{$[X]^{<\omega}$}] the set of all finite subsets of $X$
\end{description}

\chapter{The Basics}

\setcounter{section}{6}
\section{Contractions and minors}

\begin{definition}
  \label{def:minor}
  \lean{IsMinor}
  Here.
\end{definition}

\setcounter{chapter}{11}
\chapter{Graph Minors}

\section{Well-quasi-ordering}

\begin{proposition}
  \label{WQO_iff} \lean{WQO_iff} \leanok
  A quasi-ordering $\leq$ on $X$ is a well-quasi-ordering if and only if $X$
  contains neither an infinite antichain nor an infinite strictly decreasing
  sequence $x_0 > x_1 > \cdots$.
\end{proposition}

\begin{proof}
  \leanok
  Diestel's book uses Ramsey theory to prove this but in fact it is essentially
  in mathlib already.
\end{proof}

\begin{corollary}
  \label{WQO_incsubseq} \uses{WQO_iff}
  If $X$ is well-quasi-ordered, then every infinite sequence in $X$ has an
  infinite increasing subsequence.
\end{corollary}

\begin{lemma}
  \label{WQO_finset} \lean{WQO_Finset}
  If $X$ is well-quasi-ordered by , then so is $[X]^{<ω}$.
\end{lemma}

\section{The graph minor theorem for trees}