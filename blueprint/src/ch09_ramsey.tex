\chapter{Ramsey Theory for Graphs}

\begin{definition}
  \label{defi:monochromatic} \lean{Monochromatic} \leanok

  Given a $c$-coloring of $[X]^k$, the set of all $k$-subsets of $X$, we call a
  set $Y \subseteq X$ \emph{monochromatic} if all the elements of $[Y]^k$ have
  the same color.
\end{definition}

\begin{definition}
  \label{defi:mono_graph} \lean{SimpleGraph.EdgeLabeling.isMonochromatic} \leanok

  Given a $c$-coloring of $[X]^k$, the set of all $k$-subsets of $X$, we call a
  set $Y \subseteq X$ \emph{monochromatic} if all the elements of $[Y]^k$ have
  the same color.
\end{definition}

\begin{theorem}[Theorem!9.1.2 in Diestel]
  \label{theo:ramsey_inf} \lean{ramsey912} \leanok \uses{defi:monochromatic}

  Let $k$, $c$ be positive integers, and $X$ an infinite set. If $[X]^k$ is
  colored with $c$ colors, then $X$ has an infinite monochromatic subset.
\end{theorem}

\begin{proof}
  \leanok

  Following Diestel with minor changes: We prove the theorem by induction on
  $k$, with $c$ fixed. For $k = 0$ the assertion holds vacuously, so let $k > 0$
  and assume the assertion for smaller values of $k$. Let $[X]^k$ be colored
  with $c$ colors. We shall construct an infinite sequence $X_0, X_1, \ldots$ of
  infinite subsets of $X$ and choose elements $x_i \in X_i$ with the following
  properties (for all $i$):
  \begin{enumerate}
    \item $X_{i+1} \subseteq Xi \setminus \{xi\}$;
    \item all $k$-sets $\{x_i\} \cup Z$ with $Z \in [X_{i+1}]^{k−1}$ have the
    same color, which we associate with $x_i$.
  \end{enumerate}
  We start with $X_0 := X$ and pick $x_0 \in X_0$ arbitrarily. By assumption,
  $X_0$ is infinite. Having chosen an infinite set $X_i$ and $x_i \in X_i$ for
  some $i$, we $c$-color $[X_i \setminus \{xi\}]^{k−1}$ by giving each set $Z$
  the color of $\{xi\} \cup Z$ from our $c$-coloring of $[X]^k$. By the
  induction hypothesis, $X_i \setminus \{xi\}$ has an infinite monochromatic
  subset, which we choose as $X_{i+1}$. Clearly, this choice satisfies both
  conditions above. Finally, we pick $x_{i+1} \in X_{i+1}$ arbitrarily. Since
  $c$ is finite, one of the $c$ colors is associated with infinitely many $x_i$.
  These $x_i$ form an infinite monochromatic subset of $X$.
\end{proof}

\begin{theorem}
  \label{theo:ramsey_infgraph} \lean{ramsey2} \leanok \uses{defi:mono_graph}

  Let $c$ be a positive integer, and $G$ an infinite graph. If the edges of $G$
  are colored with $c$ colors, then $G$ has an infinite monochromatic subset.
\end{theorem}

\begin{proof}
  \leanok \uses{theo:ramsey_inf}

  This is a direct consequence of Theorem~\ref{theo:ramsey_inf} for $k=2$
\end{proof}